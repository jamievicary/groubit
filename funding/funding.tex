%
\documentclass[a4paper, 12pt]{article}
\usepackage[margin=3cm]{geometry}

\begin{document}

\title{Funding Application for Interactive Quantum\\Computing Workshop \textbf{qubit.ninja}}
\author{David Reutter and Jamie Vicary
\\
Department of Computer Science, University of Oxford}
\maketitle

\section*{Introduction}

This is a funding application to support an interactive quantum computing workshop called \textbf{qubit.ninja}, designed to teach the general public about quantum computing concepts in a fun and hands-on way. We are requesting funding for design, development and production of this workshop, as well as some of the associated costs of operating this as a public engagement scheme.

\paragraph{Workshop concept.}
Each participant is given their own `qubit', an attractively-designed box with buttons and lights (see Figure X), which can be networked in a linear chain by connecting adjacent boxes with cables. Participants are guided by the Workshop Leader through some basic concepts of quantum information science---such as superposition, entanglement, teleportation and dense coding---which they are then able to try out themselves with their own qubits, by pressing colour-coded buttons and watching flashing indicator lights.

\paragraph{Public engagement bookings.}
Although currently in the concept stage, our proposed workshop already has confirmed bookings for Summer 2017.

\begin{itemize}
\item The \textbf{Hay Festival of Literature and Arts} is a huge annual festival held in Hay-on-Wye over 10 days in May and June, which sells over 200,000 event tickets each year. In recent years they have started including a small number of science-based lectures and workshops into their programme, and our application to run the \textbf{qubit.ninja} workshop has been accepted to run at  this year's festival. \textit{Requirement:} Three 90-minute workshops with 15-20 participants.

\item The \textbf{Cheltenham Science Festival} has been running since 2002, and is the premier UK\ science festival. They heard about our \textbf{qubit.ninja} workshop and approached us to request that we run it at their event on 10/11 June this year, and we have accepted. \textit{Requirement:} Three 90-minute workshops with 25-30 participants.

\item We have been asked to participate at a \textbf{Science Taster Day} school event on June 6 at the Mathematical Institute in Oxford, aimed at increasing participation among 14-15 year old girls in science and technology. 
\end{itemize}
These bookings are already confirmed, and provide guaranteed impact for this project.

\paragraph{Future sustainability.}

Teaching. Selling devices \pounds 20/each.

\paragraph{Technical background.}



 

The workshop has already been successfully accepted to run  at the Hay Festival in May 2017. The Cheltenham Science Festival then found out about it, and approached us with a request for us to also run it there in June 2017, which we have accepted. The Hay and Cheltenham Science Festival are two of the premier science festivals that run annually in the UK, yielding guaranteed impact for the small financial contribution that we are asking for.

We also believe these qubit devices will be extremely useful as teaching aids for a first course in quantum information, and the second author intends to build them into two courses that he teaches regularly, \textit{Quantum Computation and Cyber Security} and \textit{Quantum Computer Science}, both in the Computer Science department at Oxford. More broadly, we believe that these `qubits' could be sold commercially at \pounds 20/device, for teaching use within universities, and also for hobbyist market. This would then support the public engagement side of the project and allow it to financially self-sustaining.


\section*{Workshop overview}

The focus of the qubit.ninja 

Basic concept. Commitment to Hay Festival and Cheltenham Science Festival. 90 minutes, 30-40 participants, presentation + audience interaction. Research background.

Explain why this is ``Impact'', link to various Oxford statements about public engagement.

Link to main research project, NDA\ with Earth Computing.


\section*{Funding requested}

\def\arraystretch{1.5}%  1 is the default, change whatever you need
\begin{tabular}{|p{13cm}|l|}
\hline
\raggedright
\textbf{Prototyping.} Basic electrical components (breadboards, resistors, LEDs), Arduino Nano microcontrollers. & \pounds 100
\\\hline\raggedright
\textbf{Trial production run.} 10 devices (commercially-printed circuit boards, surface mount components, enclosures) & \pounds 150
\\\hline\raggedright
\textbf{Main production run.} 50 devices (commercially-printed pre-assembled circuit boards, battery packs \& batteries, enclosures, adhesive stickers for external detailing). & \pounds 600 
\\\hline\raggedright
\textbf{Support.} Professional technical assistance from the Oxford Hackspace at \pounds 60/month for February--June. & \pounds 300
\\\hline\raggedright
\textbf{Workshop display.} Two large \textbf{qubit.ninja} banners to display outside and within the workshop venue.
& \pounds 100
\\\hline\raggedright
\textbf{Web presence.} http://qubit.ninja web address.
& \pounds19
\\\hline\raggedright
\textbf{Total costs.} & \pounds 1269
\\\hline\raggedright
\textbf{Funds already secured} from Cheltenham Science Festival support fund. & \pounds 100
\\\hline\raggedright
\textbf{Funding now requested.} & \pounds 1169
\\\hline
\end{tabular}

\vspace{10pt}
\noindent
Roughly one-quarter of requested funds are for technical support from the Oxford Hackspace, a nonprofit enterprise located in Central Oxford and supported by the Oxford Science Foundation, who make available laboratory space and technical assistance through professional full-time technicians for small projects such as this. We regard this as extremely good value for money to ensure the overall success of the project.

\section*{Technical details}

Clifford quantum theory, networking technologies, ...

\end{document}
